\documentclass{article}
\usepackage[utf8]{inputenc}
\usepackage{graphicx}
\usepackage{booktabs}
\usepackage{amsmath, amsthm, amssymb, bm}
\usepackage{tikz, pgfplots}
\usepackage{float}

\title{MA4240 Report}
\author{Dishank, Chirag, Vishwanath, Datta}
\date{April 2022}

% What to do
% We will run simulations to verify CLT,
% basically, we will start with a known distribution
% the distributions that we can use are
% uniform, gaussian, cauchy, chi sq, t-distribution
% we will make samples of size 100 for each distribution
% we will make 5000 such samples
% for each sample of size 100, we will find the Xbar.
% CLT states that the distribution of sample means approximates a normal distribution as the sample size gets larger, regardless of the population's distribution.
% Now that we have 1000 values for Xbar, we will make a box plot and histogram and Q-Q plot and then visualise whether or not the distribution of Xbar looks like normal distribution
% note that Xbar will have mean as u (same as original distribution) and variance as sigma/10
%
% Hypothesis Testing
%    Alternate hypothesis
%       CLT is incorrect
%    Null Hypothesis
%       CLT is correct.
% 
% Theory: 
%   type 1 error: rejecting the null hypothesis when it is true (for population parameters)
%   type 2 error: not rejecting the null hypothesis when it is false (for population parameters)
%
% type 2 error is more significant for us here
% because if CLT is actually false, and we are modelling several things assuming it to be True.
%
% The null hypothesis suggests that Xbar follows N(u, sigma*0.1)
% We can make a QQ plot since we know the original normal distribution


\begin{document}
\maketitle

\section{Introduction}
Central Limit Theorem states that normalised sum of independent and identically distributed random variables tends towards a normal distribution, irrespective of the distribution of random variables.
\begin{align}
    Z = \lim_{n \to \infty} \left(\frac{\bar{X}_n-\mu}{\frac{\sigma}{\sqrt{n}}}\right) \label{eq:CLT}
\end{align}
In this project, we propose to verify the correctness of Central Limit Theorem by running simulations beginning with a variety of distributions covered in the course.

\section{Central Limit Theorem and empirical approximation}
While equation \eqref{eq:CLT} suggests that $n$ should be a very large number. In practice, we tend to use the theorem for $n>30$. 
\subsection{Assumptions}
\begin{itemize}
\item  The samples drawn are independent.
\item  The sample size is sufficiently large
\item  The mean and variance of the sampling distribution are finite
\end{itemize}
\subsection{Proof}
Let $X_1,X_2,\dots,X_n,\dots$ be i.i.d random variables with mean $\mu$ and variance $\sigma^2$.
The sum $X_1+X_2+X_3+\dots+X_n,$ has mean = $n\mu$ and variance $n\sigma^2$.

Now consider the random variable 
\begin{equation}
    Z_n = \frac{X_1+X_2+X_3+\dots+X_n - n\mu}{\sqrt{n\sigma^2}}
\end{equation}
which is equivalent to 
\begin{equation}
    Z_n = \sum_{i=1}^n \frac{Y_i}{\sqrt{n}}
\end{equation}
where,
\begin{equation}
    Y_i = \frac{X_i-\mu}{\sigma}\label{a}
\end{equation}
Each with mean = 0 and variance = 1.

Since $Y_i$'s are all identically distributed the characteristic equation of $Z_n$ is given as.
\begin{equation}
    \phi_{Z_n}(t) =\displaystyle \prod_{i=1}^n\phi_{Y_n}\left(\frac{t}{\sqrt{n}}\right) = \left[\phi_{Y_1}\left(\frac{t}{\sqrt{n}}\right)\right]^n
\end{equation}
The characteristic equation of $Y_1$ is given
\begin{equation}
    \phi_{Y_1}\left(\frac{t}{\sqrt{n}}\right) = \left(1-\frac{t^2}{2n} + o\left(\frac{t^2}{n}\right)\right), \;\; \left(\frac{t}{\sqrt{n}}\right)\to 0 \label{b}
\end{equation}
o is the little o notation.

Now the Characteristic equation of $z_n$ in equation \eqref{a} is 
\begin{equation}
    \phi_{Z_n}(t) = \left( 1-\frac{t^2}{2n} + o\left(\frac{t^2}{n}\right) \right)^n
\end{equation}
We know that 
$$\lim_{n\to\infty} \left(1+\frac{x}{n}\right)^n = e^x$$
When we apply $\lim_{n\to\infty}$ the equation \eqref{b} will change into 
\begin{equation}
    \lim_{n\to\infty}\phi_{Z_n}(t) =\lim_{n\to\infty}\left( 1-\frac{t^2}{2n} + o\left(\frac{t^2}{n}\right) \right)^n =  e^{\frac{-1}{2}t^2}
\end{equation}
As all the higher terms will disappear as n goes to higher values.
So, The R.H.S will be equal to Characteristic equation of $\mathcal{N}(0,1)$.
Therefore as $n\to\infty$ the distribution $Z_n$ will approach $\mathcal{N}(0,1)$.
i.e. $\frac{\sqrt{n}}{\sigma}(\Bar{X}_n - \mu) $ converges to the Normal distribution $\mathcal{N}(0,1)$.
Hence proved.

\section{Shapiro-Wilk Test}
The Shapiro–Wilk test is used to determine whether a sample $x_{1},..., x_{n}$ was drawn from a normally distributed population.\\
The test statistic(W):
\begin{align*}
    W=\frac{\left(\sum_{i=1}^{n}a_{i}x_{(i)}\right)^{2}}{\sum_{i=1}^{n}(x_{i}-\bar{x})^{2}}
\end{align*}
where
\begin{itemize}
    \item $x_{(i)}$ is $i^{th}$ order statistic i.e, $i^{th}$ smallest element in the sample.
    \item $\bar{x}$ is mean of the sample($x_{1},..,x_{n}$). 
    \item $(a_{1},a_{2},..,a_{n})=\frac{m^{\top}V^{-1}}{C}$ where $C=||V^{-1}m||$.
    \item Vector $m$ is composed of the expected values of the order statistics of independent and identically distributed random variables drawn from a typical normal distribution.
    \item $V$ is the covariance matrix of those normal order statistics
\end{itemize}
\subsection{Interpretation}
This test assumes that the population is regularly distributed. If the p value is smaller than the selected alpha level, the null hypothesis is rejected, and evidence that the data being tested are not normally distributed is found. The null hypothesis, on the other hand, cannot be rejected if the p value exceeds the set alpha threshold.
\begin{itemize}
    \item \textbf{If p $\leq$ 0.05}: then the null hypothesis can be rejected (i.e. the variable is NOT normally distributed).
    \item \textbf{If p $>$ 0.05}: then the null hypothesis cannot be rejected (i.e. the variable \textbf{MAY BE} normally distributed).
\end{itemize}

\subsection{Why it works?}
the Shapiro-Wilk is more powerful because it also takes into account the co variances between the order statistics, producing a best linear estimator of $\sigma$ from the Q-Q plot, which is then scaled by s. When the distribution is far from normal, the ratio isn't close to 1.



\section{Hypothesis}
The hypothesis can be framed as follows
\begin{align}
    & H_A\text{ : The empirical approximation of the CLT does not hold} \nonumber\\
    & H_0\text{ : The empirical approximation of the CLT holds}\nonumber
\end{align}
The significance level $\alpha$ is subject to the method of normality testing. For our case, the Shapiro-Wilk test suggests using a significance level of $0.05$. 
% This means that if we obtain a p-value less than alpha, we reject the null hypothesis. Otherwise, we fail to reject the null hypothesis.

\section{Procedure}
We have chosen 4 sampling distributions. For each distribution, we generated 1000 batches of sizes 10, 30, 50 and 100 samples from the sampling distribution. We find the sample mean of each batch and call it $\bar{X}$. From Central Limit Theorem, we know
\begin{align}
    \bar{X} \sim \mathcal{N}\left(\mu, \frac{\sigma^2}{n}\right)
\end{align}
To verify the claim, we perform normality tests, which can be classified into two parts
\begin{enumerate}
    \item Graphical Methods
    \begin{itemize}
        \item Histogram
    \end{itemize}
    \item{Frequentist tests}
    \begin{itemize}
        \item Shapiro-wilk test
    \end{itemize}
\end{enumerate}

\subsection{Distributions used}
\begin{enumerate}
    \item \textbf{Standard Normal}: The pdf of standard normal distribution is given by:
    $$f_X(x) = \dfrac{1}{\sqrt{(2\pi)}}exp\left(-\dfrac{x^2}{2}\right)$$
    The mean is 0 and standard deviation is 1. Figure \ref{normal_pdf} shows the PDF of standard normal distribution.
    \begin{figure}[H]
        \centering
        \includegraphics[scale=0.5]{images/gaussian.png}
        \caption{PDF of standard normal distribution}
        \label{normal_pdf}
    \end{figure}
    \item \textbf{Continuous uniform distribution}: Here, we have used U(0, 1). The PMF is given by $$f_X(x) = \begin{cases}1,\; 0 \le x \le 1 \\ 0,\; otherwise\end{cases}$$ 
    The mean is 0.5 and standard deviation is 0.289. Figure \ref{uni_pdf} shows the PDF of the uniform distribution.
    \begin{figure}[H]
        \centering
        \includegraphics[scale=0.5]{images/uniform.png}
        \caption{PDF of uniform distribution}
        \label{uni_pdf}
    \end{figure}
    \item \textbf{Geometric Distribution}: Unlike the other distributions that we used, this distribution is for a discrete random variable. The PMF is given by
    $$f_X(k) = (1-p)^{k-1}p,\; k=1,2,3,...$$
    In our experiments, we arbitrarily chose to use $p=0.35$. The mean is $\dfrac{1}{p}$ and the standard deviation is $\dfrac{\sqrt{1-p}}{p}$. The PMF is given in figure \ref{geom_pmf}. 
    \begin{figure}[H]
        \centering
        \includegraphics[scale=0.5]{images/geometric.png}
        \caption{PMF of geometric distribution}
        \label{geom_pmf}
    \end{figure}
    \item \textbf{Standard cauchy distribution}: The PDF is given by $$f_X(x) = \dfrac{1}{\pi (1+x^2)}$$ Neither the mean nor the standard deviation are finite. Thus CLT should not apply on this distribution. Figure \ref{cauchy_pdf} shows the PDF of standard cauchy distribution.
    \begin{figure}[H]
        \centering
        \includegraphics[scale=0.4]{images/cauchy.png}
        \caption{PDF of Cauchy Distribution}
        \label{cauchy_pdf}
    \end{figure}
\end{enumerate}

\section{Results}
For Normal distribution and uniform distribution, we found that even for the smallest selected sample size of 10, the sample mean follows Normal distribution as per Shapiro-Wilk test. Thus CLT holds true for these for any sample size greater than or equal to 10. For Geometric distribution, we found that CLT did not hold for sample sizes 10 and 30 as per Shapiro-Wilk test. However, CLT held true for sample sizes 50 and 100. For Cauchy distribution, we found that CLT did not hold for any sample size. 

\section{Conclusion}
While using CLT, the empirical approximation is a good one but it may fail. To get better results, one may want to consider using a larger sample size of 50. Also, one may want to verify that the sampling distribution has finite mean and sample variance before applying CLT.

\section{Contribution}
\begin{itemize}
    \item Chirag: Sections 1, 4, 5 and Code
    \item Dishank: Sections 5, 6, 7 and Code
    \item Savarana Datta: Section 3 and Slides
    \item Vishwanath: Section 2 and Slides
\end{itemize}
\end{document}
